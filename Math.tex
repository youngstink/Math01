\documentclass{article}
\usepackage{amsfonts,amssymb}
\usepackage[margin=1in]{geometry}

\title{CSCI 301, Winter 2017\\Math Exercises \# 1}
\author{YOUR NAME HERE}
\date{Due date:  Monday, January 23, midnight.}

\begin{document}

\maketitle

\begin{description}
\item[Exercises for Section 1.1]
\item[C.] Find the following cardinalities
\item[30.] $|\{\{1,4\},a,b,\{\{3,4\}\},\{\emptyset\}\}| = 5$

\item[Exercises for Section 1.3]
\item[A.] List all the subsets of the following sets.
\item[8.] $\{\{0,1\},\{0,1,\{2\}\},\{0\}\}$ 
\item $\{\emptyset\},\{0,1\},\{0,1\{2\}\},\{0\},\{\{0,1\},\{0,1\{2\}\}\},\{\{0,1\},\{0\}\},\{\{0,1\{2\}\},\{0\}\},\{\{0,1\},\{0,1,\{2\}\},\{0\}\}$

\item[Exercises for Section 1.4]
\item[A.] Find the indicated sets.
\item[12.] $\{X\in\mathcal{P}(\{1,2,3\}):2\in X\}$
\item $\{2\},\{1,2\},\{2,3\},\{1,2,3\}$

\item[B.] Suppose that $|A|=m$ and $|B|=n$.  Find the following cardinalities.
\item[18.] $|\mathcal{P}(A\times \mathcal{P}(B))|$
\item[Step 1:] Note that the cardinality of a power set $\mathcal{P}(A)$ is equal to $2^x$ where x is equal to $|A|$.
\item[Step 2:]$\mathcal{P}(B) = 2^n$.
\item[Step 3:]Now we look at the set $A\times \mathcal{P}(B)$. This Cartisian product is equal to $\{(a,b): a\in A,b\in \mathcal{P}(B)\}$ therefore its cardinality is equal to $m(2^n)$.
\item[Solution]$|\mathcal{P}(A\times \mathcal{P}(B))| = 2^k$ where $k = m(2^n)$


\item[Exercises for Section 2.10]  Negate the following sentences.

\textit{}
\item[8.] If $x$ is a rational number and $x\not = 0$, then
  $\tan(x)$ is not a rational number.
  \item $((x\in\mathbb{Q} )\land (x \not= 0))\rightarrow (\tan(x) \not\in\mathbb{Q} )$
  \item $\neg((x\in\mathbb{Q} )\land (x \not= 0))\rightarrow \neg(\tan(x) \not\in\mathbb{Q} )$
  \item $((x\not\in\mathbb{Q} )\land (x \not= 0))\rightarrow (\tan(x) \in\mathbb{Q} )$
\item If $x$ is not a rational number and $x\not = 0$, then $\tan(x)$ is a rational number.
  
\item[Exercises for Section 3.1]

\item[4.]  Five cards are dealt off a standard 52-card deck and
  lined up in a row.  How many such lineups are there in
  which all 5 cards are of the same suit?
\item $13$ to the $5$ falling factorial powers is equal to $13 * 12 * 11 * 10 * 9 = 154440$ different line ups of 5 cards in the same suit with no repeats.

\item[Exercises for Section 2.5] Write a truth table for the logical
  statements in problems 1--9:

\item[8.] $P \vee (Q \wedge \neg R)$

\begin{table}
\caption{Truth Table for Ex.2.5}
\centering
\begin{tabular}{|l|l|l|l|}
\hline
$P$ & $Q$ & $R$ & $P \vee (Q \wedge \neg R)$ \\
\hline
T & T& T& T\\
\hline
T & T& F& T\\
\hline
T & F & T & T\\
\hline
T & F & F & T\\
\hline
F & T& T & F\\
\hline
F & T& F & T\\
\hline
F & F & T & F\\
\hline
F & F & F& F\\\hline
\end{tabular}
\end{table}

\item[Exercises for Section 3.2]

\item[8.] Compute how many 7-digit numbers can be made from the digits
  1,2,3,4,5,6,7 if there is no repetition and the odd digits
  must appear in an unbroken sequence.  (Examples: 3571264 or 2413576 or
  2467531, but not 7234615.)
\item In a series of 7-digits with no repeats where all numbers are included and the odd numbers appear in an unbroken sequence, there are four possible arrangements possible of odd and even numbers. Even numbers- e odd numbers- o : ooooeee, eooooee, eeooooe, eeeoooo. There are $4!$ possible variations of odd numbers and for each of those variations there are $3!$ variations of even numbers. Therefore the number of 7-digit numbers that can be made from the digits
  1,2,3,4,5,6,7 if there is no repetition and the odd digits
  must appear in an unbroken sequence is equal to:
  \item $4\times(4!\times 3!) = 576$ different numbers

\item[Exercises for Section 3.3]

\item[12.] Twenty-one people are to be divided into two teams, the Red
  Team and the Blue Team.  There will be 10 people on the Red Team and
  11 people on the Blue Team.  In how many ways can this be done?
\item For this problem we can focus on one team to find how many possible combinations of players there are because the order of the second team doesn't matter. We will treat the second team as the team that is rejected from the first team. Because the order doesn't matter in this problem the number of possibilities you can make a team of 11 out of 21 people if equal to 11 falling factorial factors of 21.
\item $21\times 20\times 19\times  18\times 17\times 16\times 15\times 14\times 13\times 12\times 11 = 14079294028800$ player combinations.
\item[Exercises for Section 3.5]


\item[8.] This problem concerns 4-card hands dealt off of a standard
  52-card deck.  How many 4-card hands are there for which all 4 cards
  are of different suits or all 4 cards are red?
\item There are $|A|$ hands where A is the set of hands where all 4 cards are a different suit. There are $|B|$ hands where B is the set of hands with all red cards.
\item $|A| = (52\times 39\times 26\times 13)/4 = 171366$
\item $|B| = 26!\div 22! = 35880$
\item These two sets have nothing in common because you cannot have all red cards of 4 different suits. There for the total number of hands is equal to $|A|+|B|$
\item$|A|+|B| = 207246$ different hands with either 4 different suits or all red cards. 



\end{description}
\end{document}
